\label{sec:Intro}
\section*{Introduction}

Scientific consensus has been reached about the rapid environmental change we are experiencing, a rise in temperature of more than 0.8°C in the 1901-2012 period~\citep{stocker_ipcc_2013}. But high inter-annual variability of climate makes temperatures fluctuate around the increasing trend.

The increase of temperature caused several organisms to advance in their phenology, i.e. the timing of major events in their life cycle. There is evidence of trend to precocity in plants in their flowering time and bud-burst date~\citep{alberto_adaptive_2011, gienapp_predicting_2013}. Natural populations have three ways to react to environmental changes: extinction, adaptation (genetic adaptation and plasticity) or migration.

Quantitative genetics models have investigated for several decades how species would react to changes in their environment~(reviewed in \citealt{burger_quantitative-genetic_2004}). They can predict the evolution of phenotype in populations by modeling reproduction and selection processes. Studied traits are often modeled as follow: in a given environment there is an optimal trait value, population then converge to this value to maximize fitness if achievable~\citep{lande_quantitative_1982}.
Long-lived plant species experience selection differently from annual plants because of their particular stage structure: seedlings and grown trees may not experience the same selection pressures, especially  under a warmer climate~\citep{lande_quantitative_1982, coulson_dynamics_2008, barfield_evolution_2011, engen_evolution_2011}.

There has been an active development of models for age-structured population, but theoretical results on stage-structured population are more recent~\citep{barfield_evolution_2011}. Most environmental changes model have been studied through optimum variations either with a trend in the evolution of the optimum or with random fluctuations around a mean, but trend with fluctuations have been overlooked in the literature.

Here we focused on the effect of fluctuations on a stage-structured tree population, with bud-burst date as the evolving trait and two different optima — one for each stage.
We used a previously developed demographic and quantitative genetics models (see~\nameref{sec:M&M}), and implemented environmental fluctuations, neglecting phenotypic plasticity. The sessile oak (\textit{Quercus petraea} spp.) is a widespread European species, with we well-known phenology in natural populations~\citep{alberto_adaptive_2011}. Using a process-based model describing the phenology and life cycle of the sessile oak we estimated those fluctuations in optimal trait values.