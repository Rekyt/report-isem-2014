\label{sec:Intro}
\section*{Introduction}

Scientific consensus has been reached about the rapid environmental change we are experiencing, a rise in temperature of more than 0.8°C in the 1901-2012 period~\citep{stocker_ipcc_2013}. But high inter-annual variability of climate makes temperature fluctuate around the increasing trend.

The increase of temperature caused several organisms to advance in their phenology, i.e. the timing of major events in their life cycle. There is evidence of trend to precocity in plants in their flowering time and bud-burst date~\citep{alberto_adaptive_2011, gienapp_predicting_2013}. Natural populations have three ways to react to environmental changes: extinction, genetic adaptation (including plasticity) or migration.

As long-lived species trees experience selection differently from annual plants because of their particular stage structure: seedlings and grown trees do not experience the same selection pressures~\citep{lande_quantitative_1982, coulson_dynamics_2008, barfield_evolution_2011, engen_evolution_2011}.

Quantitative genetics models have investigated for several decades how would species react to changes in their environment~(reviewed in \citealt{burger_quantitative-genetic_2004}). They can predict the evolution of phenotype across population modeling reproduction and selection processes.

Models generally take into account a single trait with a single optimal value maximizing the fitness of an individual~\citep{lande_quantitative_1982}. To mimic variable environments they either vary the optimum in a linear fashion or randomly fluctuate it around a given mean~\citep{lande_role_1996}.

There have been an active development of models for age-structured population, but theoretical results on stage-structured population were only shown recently~\citep{engen_evolution_2011}. In trees, a stage-structure view of the population may be more relevant than an age-structured one, as there is no strong differences between a tree of 1 or 2 years old. Most environmental changes model have been studied either with a trend in the evolution of the environment or with random fluctuations around a mean, but trend with fluctuations have been overlooked in the literature.

Here we focused on the effect of fluctuations on a stage-structured tree population, with bud-burst date as the evolving trait and two different optima — one for each stage.
We used a previously developed demographic and quantitative genetics models (see~\autoref{sec:M&M}), and implemented environmental fluctuations. Using a process-based model describing the phenology and life cycle of Oak trees we estimated those fluctuations.