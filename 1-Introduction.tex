\label{sec:Intro}
\section*{Introduction}

Scientific consensus has been reached for several years about the rapid environmental change we are experiencing, a rise of more than 0.8°C in the 1901-2012 period~\citep{stocker_ipcc_2013}. But, the high interannual variability makes temperature fluctuating around the increasing trend.

The increase of temperature caused several organisms to advance in their phenology, i.e. the timing of various seasons. There have been evidences of trend to precocity in plants in their flowering time and bud-burst date~\citep{alberto_adaptive_2011, gienapp_predicting_2013}.

Models generally take into account a single trait with a single optimal value maximizing the fitness of an individual~\citep{lande_quantitative_1982}. To mimic variable environments they either vary the optimum in a linear fashion or randomly fluctuate it around a given mean~\citep{lande_role_1996}.

As long-lived species trees experience selection differently from annual plants, they buffer selection across several years, because of their particular stage structure: seedlings and grown trees do not experience the same selection pressures~\citep{lande_quantitative_1982, coulson_dynamics_2008, barfield_evolution_2011, engen_evolution_2011}. 

Here we focused on the effect of fluctuations on a stage-structured tree population, with bud-burst date as the observed trait, and two different optima one from each stage.
We used a previously developed demographic and quantitative genetics models (see \ref{sec:M&M}), and implemented environmental fluctuations. Using \textsc{PHENOFIT}'s simulations we estimated those fluctuations in the wild.