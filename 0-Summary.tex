\section*{Summary}

Because of climate change we observe a general advance of phenology in various species. Environmental fluctuations are also known to have great consequences on adaptation dynamics. Quantitative genetics models have been developed to predict those dynamics, focusing on quantitative traits with optimal values. None has investigated yet the impact of climate change with fluctuations on long-lived species such as tree.
We model a stage-structured tree population in two stages: mature and immature individuals, each class having an optimal bud-burst value influencing their fitness components. We then vary optima with and without fluctuations.
Fluctuations seem to decrease survival in our population, but as we focused on long-lived species there seem to be a buffer effect of our classes: they do not react very much to fluctuations. From \textsc{PHENOFIT} simulations we estimated a variation of optimal bud-burst date for fecundity of -0.15 day per year during the next century, we also showed a increase frequency of extreme events where all fecundities are equal to 0.
However, our model does not take into account phenotypic plasticity which may decreases the observed genetic evolution.