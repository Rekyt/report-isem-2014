\section*{Summary}

Adapting to a warmer climate represents a challenge for long-lived organisms such as trees. Climate also fluctuates from year to year, with consequences for adaptive dynamics. Quantitative genetics models have been developed to predict those dynamics, focusing on quantitative traits with optimal values. The impact of optima fluctuations on long-lived  stage-structured species has not been investigated yet.
We model a stage-structured tree population with two stages: immature and mature individuals, each class having an optimal bud-burst date influencing respectively their fecundity and survival. Optimal dates vary from year to year either normally around a given mean, or they fluctuate around a decreasing trend with time.
Fluctuations decrease survival, but the age structure of our population buffers the effect of fluctuating climate on the genetic composition of the population. From simulations of a process-based model of tree phenology, we estimated a trend of advance of optimal bud-burst date of -0.15 day per year during the next century, and an increased frequency of extreme events where all fecundities are equal to 0. Our model was parametrized according to the phenology of the sessile oak (\textit{Quercus petraea}), from our simulations the population will halve in the next 150 years but will go extinct.