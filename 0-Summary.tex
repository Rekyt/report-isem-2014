\section*{Summary}

Adapting to a warmer climate represents a challenge for long-lived organisms such as trees. Climate also fluctuates from year to year, with consequences for adaptive dynamics. Quantitative genetics models have been developed to predict those dynamics, focusing on quantitative traits with optimal values. The impact of climate change with fluctuations on long-lived species has not been investigated yet.
We model a stage-structured tree population with two stages: mature and immature individuals, each class having an optimal bud-burst date influencing respectively their fecundity and survival. Optimal dates vary from year to year and advance as climate warms.
Fluctuations decrease survival, but the age structure of our population buffers the effect of fluctuating climate on the genetic composition of the population. From simulations of a process-based model of tree phenology and life cycle, we estimated a trend of advance of optimal bud-burst date of -0.15 day per year during the next century, we also showed a increase frequency of extreme events where all fecundities are equal to 0.