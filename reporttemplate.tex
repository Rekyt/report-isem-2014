\documentclass[a4paper, 12pt]{article}

%% Specify margins
\usepackage[margin=2cm]{geometry}

%% Use specific font
\usepackage{fontspec}
\setmainfont{Times New Roman}

\usepackage{mathtools}

%% ** Colors **
\usepackage{color} 
\usepackage[table]{xcolor}
\definecolor{linkcol}{RGB}{0,0,238}
\definecolor{lightgray}{gray}{0.8}

%% For hyperref (link for figures, equations and papers)
\usepackage{hyperref}
\hypersetup{
    breaklinks,
    baseurl       = http://,
    colorlinks    = true,
    linkcolor     = linkcol,
    citecolor     = linkcol,
    pdfborder     = 0 0 0
}

%% Bibliography
\usepackage{natbib}

%% Get label of figures bold
\usepackage[labelfont=bf]{caption}

% Write units properly
\usepackage{siunitx}
\DeclareSIUnit\year{yr}

\begin{document}
\renewcommand{\arraystretch}{1.5}

%TC:ignore
\section*{Summary}

Adapting to a warmer climate represents a challenge for long-lived organisms such as trees. Climate also fluctuates from year to year, with consequences for adaptive dynamics. Quantitative genetics models have been developed to predict those dynamics, focusing on quantitative traits with optimal values. The impact of climate change with fluctuations on long-lived species such as tree has not been investigated yet.
We model a stage-structured tree population with two stages: mature and immature individuals, each class having an optimal bud-burst date influencing respectively their fecundity and survival. Optimal dates vary from year to year and advance as climate warms.
Fluctuations decrease surviva, but the age structure of our population buffers effect of fluctuating climate on the genetic composition of the population. From simulations of a process-based model of tree phenology and life cycle we estimated a variation of optimal bud-burst date of -0.15 day per year during the next century, we also showed a increase frequency of extreme events where all fecundities are equal to 0.
%TC:endignore
\section*{Introduction}

\begin{itemize}
	\item climate change
	\item adaptation, phenotypic plasticity, predict adaptation
	\item tree demography success or extinction?
\end{itemize}

\label{sec:M&M}
\section*{Materials and Methods}

\subsection*{Population model}

We used a previously developed model with stage-structure (Cite XXXXX Sandell et al.). We have two classes in our simulated tree population: immatures (I) and matures (M).
Explain the parameters.
The corresponding Lefkovitch matrix is:
\begin{equation}
	\label{eq:popmat}
	A =
	\begin{pmatrix}
	a_{II} & a_{IM} \\
	a_{MI} & a_{MM}
	\end{pmatrix}
	=
	\begin{pmatrix}
	s_{0} m f_{1} + s_{I} (1 - m) & s_{0} f_{2} \\
	s_{M} m & s_{M}
	\end{pmatrix}
\end{equation}

From the original model (Sandell et al. 2014) we implemented density-dependence, so that population will not continuously increase but reach a plateau (see Fig.~\ref{fig:dd}). We chose to implement density-dependence through seed germination and survival parameter $s_{0}$ using a Beverton-Holt function to avoid chaotic behaviors:
\begin{equation}
	\label{eq:ddfunc}
	s_{0} = \frac{s_{0, max}}{1 + k_{I} N_{I} + k_{M} N_{M}}
\end{equation}

with $k_{I}$ and $k_{M}$ the weights of immature ($N_{I}$) and mature ($N_{M}$) population respectively. $s_{0, max}$ is the maximum achievable $s_{0}$.

\subsection*{Life-history traits}

We considered certain life-history trait $s_{I}, f_{1}, f_{2}$ as gaussian for each individual such as:

\begin{equation}
	\label{eq:indlht}
	s_{I}(z) = s_{I}(\theta_{s})	\exp\left(-\frac{(z - \theta_{s})^2}{2\omega_{s}}\right)
\end{equation}

Averaging over the population it gives:

\begin{equation}
	\label{eq:poplht}
	\overline{s_{I}}(\overline{z_{I}}) = s_{I}(\theta_{s}) \sqrt{\frac{\omega_{s}}{\omega_{s}+P_{I}}}	\exp\left(-\frac{(\overline{z_{I}} - \theta_{s})^2}{2(\omega_{s}+P_{I})}\right)
\end{equation}

\subsection*{Iterations at each timestep}

Assuming the phenotype has a Gaussian distribution,  the mean genotypic value of matures and immatures at the next timestep is given by (\citealt{barfield_evolution_2011} Eq.5) :

\begin{align}
		\label{eq:genotypic}
		\overline{g_{I}}' &= (c_{I M} \overline{g_{M}} + c_{I I} \overline{g_{I}}) 
			\left(c_{I M} G_{M} \frac{\partial \log \overline{a_{IM}}}{\partial \overline{z_{M}}} + c_{I I} G_{I} \frac{\partial \overline{a_{I I}}}{\partial \overline{z_{I}}} \\
		\overline{g_{M}}' &=	 (c_{M I} \overline{g_{I}} + c_{M M} \overline{g_{M}}) 
			\left(c_{M I} G_{I} \frac{\partial \log \overline{a_{M I}}}{\partial \overline{z_{I}}} + c_{M M} G_{M} \frac{\partial \overline{a_{M M}}}{\partial \overline{z_{M}}}
\end{align}

When there is a reproduction event, the phenotype of the newborn is computed as such:
...

\subsection*{Approximation under weak selection}

From \citep{engen_evolution_2011} and Sandell et al. we get the following approximation of the mean phenotype in the population:
\begin{equation}
	z = ...
\end{equation}

\subsection*{Fluctuating environment}

To mimic a fluctuating environment, the optimums are fluctuating in various ways around a mean.

Under fluctuations we get another approximation supposing weak selection \citep{engen_evolution_2011}:
...

\subsection*{Trend in change}
...

\subsection*{Phenofit data}

PHENOFIT is phenology model...

On 6 localities (see map .) we had modelled budburst date and predicted fitnesses $\pm$ 21 days around this date, from these data we predicted the optimums fluctuations:
...
All statistical analysis were made using R, for the plots we used the package ggplot2.

\begin{table}
\begin{center}
	\rowcolors{1}{white}{lightgray}
	\begin{tabular}{l c c}
		\hline \hline
		Parameter & Notation & Value \\
		\hline
		\multicolumn{3}{l}{\textbf{Life Cycle}} \\
		Optimal phenotype for fecundity & $\theta_{f}$ & 100 \\
		Optimal phenotype for immature survival & $\theta_{s}$ & 130 \\
		Fecundity function width & $\omega_{f}$ & 400 \\
		Survival function width & $\omega_{s}$ & 400 \\
		Heritability & $h^2$ & 0.5 \\
		Phenotypic variance of immatures & $P_{I}$ & 40 \\
		Phenotypic variance of matures & $P_{M}$ & 40 \\
		Genotypic variance of immatures & $G_{I} = P_{I} \times h^2$ & 20 \\
		Genotypic variance of matures & $G_{M}$ & 20 \\
		Survival of immature at phenotypic optimum & $\overline{s_{I}}(\overline{z} = \theta_{s})$ & 0.8 \\
		Fecundity of first time reproducers at optimum & $\overline{f_{1}}(\overline{z} = \theta_{f})$ & 100 \\
		Fecundity of experienced reproducers at optimum & $\overline{f_{2}}(\overline{z} = \theta_{f})$ & 200 \\
		Maturation rate of immature & $m$ & 0.02 \\
		Combined survival and germination rate of seed & $s_{0}$ & 0.03 \\
		Survival of mature stage & $s_{M}$ & 0.99 \\
		\multicolumn{3}{l}{\textbf{Density-dependence}} \\
		Maximum $s_{0}$ in density-dependence function & $s_{0, max}$ & 0.12 \\
		Decreasing factor due to immatures & $k_{I}$ & 0.0001 \\
		Decreasing factor due to matures & $k_{M}$ & 0.005 \\
		\multicolumn{3}{l}{\textbf{Fluctuations}} \\
		Sensitivity of optimum for fecundity to fluctuation & $\alpha_{f}$ & 5 \\
		Sensitivity of optimum for survival to fluctuation & $\alpha_{s}$ & 5 \\
		Noise variance for fecundity & $\sigma_{\xi_{f}}^2$ & 3.725 \\
		Noise variance for survival & $\sigma_{\xi_{s}}^2$ & 3.725 \\
		Correlation between noises & $\rho_{N}$ & 0.5 \\
		\hline \hline
	\end{tabular}
	\caption{Standard parameter set}
	\label{tab:params}
\end{center}
\end{table}

%Basic explanation of the models. We modeled a stage-structured population in two stages: immatures and matures. The demography is given by a transition matrix, with...
%
%\subsection*{Under constant environment, no plasticity}
%
%Using \citet{lande_adaptation_2009}, under weak selection we have:
%\begin{equation}
%	\label{eq:dz}
%	\Delta\overline{z} = \frac{d\ln\overline{\lambda}(\overline{z})}{d\overline{z}} = \frac{1}{\overline{\lambda}(\overline{z})} \frac{d\overline{\lambda}(\overline{z})}{d\overline{z}}
%\end{equation}
%
%And we have:
%\begin{align*}
%	\overline{\lambda}(\overline{z}) &= \sum_{i,j}{v_{i} u_{j} \overline{a_{ij}}} \\
%	&= v_{I} u_{I} \overline{a_{II}} + v_{I} u_{M} \overline{a_{IM}} + v_{M} u_{I} \overline{a_{MI}} + v_{M} u_{M} \overline{a_{MM}}
%\end{align*}
%
%With $\overline{a_{ij}}$ the expected values of the coefficent of the transition matrix. Thus,
%\begin{align}
%	\overline{\lambda}(\overline{z}) &= v_{I} u_{I} \left[ \overline{f_{1}}(\overline{z}) m s_{0} + (1-m) \overline{s_{I}}(\overline{z}) \right] + v_{I} u_{M} s_{0} \overline{f_{2}}(\overline{z}) \nonumber \\
%	&\quad + v_{M} u_{I} m s_{M} + v_{M} u_{M} s_M \\
%	\label{eq:dlambda}
%	\frac{d\overline{\lambda}(\overline{z})}{d\overline{z}} &= v_{I} u_{I} \left[ \frac{d\overline{f_{1}}(\overline{z})}{d\overline{z}} m s_{0} + (1-m) \frac{d\overline{s_{I}}(\overline{z})}{d\overline{z}} \right] + v_{I} u_{M} s_{0} \frac{d\overline{f_{2}}(\overline{z})}{d\overline{z}}
%\end{align}
%
%Because $f_{i}$ and $s_{I}$ are gaussians we can write the population means $\overline{f_{i}}$ and $\overline{s_{I}}$ easily.
%
%\begin{subequations}
%	\begin{align}
%	\label{eq:meanf}
%		\overline{f_{1}}(\overline{z}) &= f_{1}(\theta_{f}) \sqrt{\frac{\omega_{f}}{\omega_{f} + P_{I}}} \exp\left(-\frac{(\overline{z}-\theta_{f})^2}{2(\omega_{f}+P_{I})}\right) \\
%		\overline{f_{2}}(\overline{z}) &= f_{2}(\theta_{f}) \sqrt{\frac{\omega_{f}}{\omega_{f} + P_{M}}} \exp\left(-\frac{(\overline{z}-\theta_{f})^2}{2(\omega_{f}+P_{M})}\right) \\
%		\overline{s_{I}}(\overline{z}) &= s_{I}(\theta_{s}) \sqrt{\frac{\omega_{s}}{\omega_{s} + P_{I}}} \exp\left(-\frac{(\overline{z}-\theta_{s})^2}{2(\omega_{s}+P_{I})}\right)
%	\end{align}
%\end{subequations}
%
%Thus we can derive these expression with respect to $\overline{z}$:
%
%\begin{align}
%	\label{eq:dfdz}
%	\frac{\partial \overline{f_{1}}(\overline{z}) }{ \partial \overline{z} } &= f_{1}(\theta_{f}) \sqrt{\frac{\omega_{f}}{\omega_{f} + P_{I}}} \frac{\partial \exp \left(-\frac{(\overline{z}-\theta_{f})^2}{2(\omega_{f}+P_{I})}\right)}{\partial\overline{z}} \nonumber \\
%	&= f_{1}(\theta_{f}) \sqrt{ \frac{\omega_{f}}{ \omega_{f} + P_{I}}}\exp\left(-\frac{(\overline{z}-\theta_{f})^2}{2(\omega_{f}+P_{I})}\right) \frac{\theta_{f} - \overline{z}}{\omega_{f} + P_{I}} \nonumber \\
%	&= \overline{f_{1}}(\overline{z}) \frac{\theta_{f} - \overline{z}}{\omega_{f} + P_{I}}
%\end{align}
%
%We obtain similar formulas for $\overline{f_{2}}$ and $\overline{s_{I}}$. Plugging \eqref{eq:dfdz} into \eqref{eq:dlambda} we have:
%
%\begin{equation}
%	\label{eq:finaldlambda}
%	\frac{d\overline{\lambda}(\overline{z})}{d\overline{z}} = v_{I} u_{I} \left[ \frac{\theta_{f} - \overline{z}}{\omega_{f} + P_{I}} m s_{0} + (1-m) \frac{\theta_{s} - \overline{z}}{\omega_{s} + P_{I}} \right] + v_{I} u_{M} s_{0} \frac{\theta_{f} - \overline{z} }{\omega_{f} + P_{M}}
%\end{equation}
%
%Using \eqref{eq:finaldlambda} into \eqref{eq:dz} gives us after rearranging:
%We have for variations of phenotype, under weak selection:
%\begin{equation}
%	\label{eq:cstdeltaz}
%	\Delta\overline{z} = 
%		(\theta_{f} - \overline{z})
%		\left[ \frac{ v_{I} u_{I} G_{I} s_{0} m \overline{f_{1}} }{ \lambda (P_{I}+\omega_{f}) }
%			+ \frac{ v_{I} u_{M} G_{M} s_{0} \overline{f_{2}} }{ \lambda (P_{M} + \omega_{f}) }
%		\right]
%		+ (\theta_{s} - \overline{z})
%		\left[ \frac{ v_{I} u_{I} G_{I} \overline{s_{I}} (1-m) }{ \lambda (P_{I}+\omega_{s}) }
%		\right]
%\end{equation}
%
%Within the square brackets, we see weighting average of fecundity and survival. Thus, we define them as $\gamma_{f}$ and $\gamma_{s}$ such as:
%
%\begin{subequations}
%	\begin{equation}
%	\label{eq:gammaf}
%	\gamma_{f} = \frac{v_{I} u_{I} s_{0} m \overline{f_{1}} }{\lambda(P_{I}+\omega_{f})} + \frac{ v_{I} u_{M} \frac{G_{M}}{G_{I}} s_{0} \overline{f_{2}}}{\lambda ( P_{M} + \omega_{f} )}
%	\end{equation}
%	and
%	\begin{equation}
%	\label{eq:gammas}
%	\gamma_{s} = \frac{ v_{I} u_{I} \overline{s_{I}} (1-m) }{\lambda(P_{I}+\omega_{s})}
%	\end{equation}
%\end{subequations}
%
%We end up having a simpler expression for $\Delta\overline{z}$ under constant environment:
%
%\begin{align}
%	\Delta\overline{z} &= -G_{I} \left[ \gamma_{f}(\overline{z} - \theta_{f}) + \gamma_{s}(\overline{z} - \theta_{s}) \right] \nonumber \\
%	\Delta\overline{z} &= - G_{I} \gamma(\overline{z} - \theta_{v})
%\end{align}
%
%with
%\begin{align}
%	\label{eq:gamma}
%	\gamma &= \gamma_{f} + \gamma_{s} \\
%	\label{eq:thetav}
%	\theta_{v} &= \frac{\frac{\gamma_{f}}{\gamma_{s}}\theta_{f} + \theta_{s}}{\frac{\gamma_{f}}{\gamma_{s}} + 1}
%\end{align}
%
%\subsection*{Under varying environment, without plasticity}
%From \citet{engen_evolution_2011}, we derived equations for mean variation of phenotype on our model.
%
%We supposed an auto-correlated fluctuating environment $\epsilon_{t}$ influencing optimums $\theta_{i}$ such as:
%\begin{align}
%	\label{eq:epstheta}
%\left\{
%	\begin{aligned}
%		\theta_{i}(t) &= \overline{\theta}_{i} + \alpha_{i}\epsilon_{t}\\
%		\epsilon_{t+1} &= (1-\rho)\overline{\epsilon} + \rho\epsilon_{t} + \xi
%	\end{aligned}
%\right.
%\end{align}
%with $\alpha_{i}$ the dependence factor of the optimum on the environment, $\rho$ the auto-correlation coefficient of the environment, $\overline{\epsilon}$ the expected environment and $\xi$ a gaussian noise vector with variance $\sigma^{2}_{\xi}$ and mean $0$. We chose $\overline{\epsilon}=0$ to simplify the calculations so that $\epsilon_{t+1} = \rho\epsilon_{t} + \xi$, we can see that:
%
%\begin{align}
%	\theta_{i}(t+1) &= \overline{\theta}_{i} + \alpha_{i}\epsilon_{t+1} \nonumber \\
%	&= \overline{\theta}_{i} + \alpha_{i}(\rho\epsilon_{t} + \xi) \nonumber \\
%	&= \overline{\theta}_{i} + \alpha_{i}\rho(\frac{\theta_{i}(t)-\overline{\theta}_{i}}{\alpha_{i}}) + \alpha_{i}\xi \nonumber \\
%	\label{eq:thetait}
%	\theta_{i}(t+1) &= \overline{\theta}_{i}(1-\rho) + \rho\theta_{i}(t) + \alpha_{i}\xi
%\end{align}
%
%The auto-correlation in the environment $\epsilon_{t}$ causes $\theta_{i}$ to be auto-correlated with the same correlation coefficient $\rho$.
%
%Using the same approach as in a constant environment, under weak selection, we end up having a similar equation than \eqref{eq:cstdeltaz} but with optimum depending on environment:
%
%\begin{equation}
%	\label{eq:deltazt}
%	\Delta\overline{z}_{t} = 
%		(\theta_{f}(t) - \overline{z_{t}})
%		\left[ \frac{ v_{I} u_{I} G_{I} s_{0} m \overline{f_{1}} }{ \lambda_{t} (P_{I}+\omega_{f}) }
%			+ \frac{ v_{I} u_{M} G_{M} s_{0} \overline{f_{2}} }{ \lambda_{t} (P_{M} + \omega_{f}) }
%		\right]
%		+ (\theta_{s}(t) - \overline{z_{t}})
%		\left[ \frac{ v_{I} u_{I} G_{I} \overline{s_{I}} (1-m) }{ \lambda_{t} (P_{I}+\omega_{s}) }
%		\right]
%\end{equation}
%
%Plugging \eqref{eq:epstheta} in \eqref{eq:deltazt} we obtain
%
%\begin{subequations}
%	\begin{align}
%		\Delta\overline{z}_{t} &= 
%			(\overline{\theta}_{f} + \alpha_{f}\epsilon_{t} - \overline{z_{t}})
%			\left[ \frac{ v_{I} u_{I} G_{I} s_{0} m \overline{f_{1}} }{ \lambda_{t} (P_{I}+\omega_{f}) }
%				+ \frac{ v_{I} u_{M} G_{M} s_{0} \overline{f_{2}} }{ \lambda_{t} (P_{M} + \omega_{f}) }
%			\right]
%			+ (\overline{\theta}_{s} + \alpha_{s}\epsilon_{t} - \overline{z_{t}})
%			\left[ \frac{ v_{I} u_{I} G_{I} \overline{s_{I}} (1-m) }{ \lambda_{t} (P_{I}+\omega_{s}) }
%			\right] \nonumber \\
%	\end{align}
%	Defining the same $\gamma_{f}$, $\gamma_{s}$ and $\theta_{v}$ as in \eqref{eq:gammaf}, \eqref{eq:gammas} and \eqref{eq:thetav}, respectively:
%	\begin{align}
%			\Delta\overline{z}_{t} &= G_{I} \left[(\overline{\theta}_{f} + \alpha_{f}\epsilon_{t} - \overline{z_{t}})
%			\gamma_{f}
%			+ (\overline{\theta}_{s} + \alpha_{s}\epsilon_{t} - \overline{z_{t}})
%			\gamma_{s} \right] \nonumber \\
%			&= - G_{I} \left[ \gamma_{f}(\overline{z_{t}} - \theta_{f}) + \gamma_{s}(\overline{z_{t}} - \theta_{s}) \right] -G_{I} \epsilon_{t} \left( \gamma_{f}\alpha_{f} + \gamma_{s}\alpha_{s} \right) \nonumber \\
%			\label{eq:deltaztdeltaz}
%			\Delta\overline{z}_{t} &= \Delta\overline{z} + G_{I} \epsilon_{t} \left( \gamma_{f}\alpha_{f} + \gamma_{s}\alpha_{s} \right) \\
%			&= -G_{I}\gamma(\overline{z_{t}} - \theta_{v})  + G_{I} \epsilon_{t} \left( \gamma_{f}\alpha_{f} + \gamma_{s}\alpha_{s} \right) \nonumber \\
%			&= -G_{I}\gamma \left(\overline{z_{t}} - \theta_{v} - \epsilon_{t} \frac{\gamma_{f}\alpha_{f}+\gamma_{s}\alpha_{s}}{\gamma} \right) \nonumber \\
%			\label{eq:totaldeltazt}
%			\Delta\overline{z}_{t} &= -G_{I}\gamma (\overline{z_{t}} - \theta_{v} - \alpha_{v}\epsilon_{t})
%	\end{align}
%\end{subequations}
%With $\alpha_{v}$ a weighted component between $\alpha_{f}$ and $\alpha_{s}$, defined in similar fashion as $\theta_{v}$ in \eqref{eq:thetav}:
%\begin{align}
%	\alpha_{v} &= \frac{\gamma_{f}\alpha_{f}+\gamma_{s}\alpha_{s}}{\gamma} \nonumber \\
%	&= \frac{ \gamma_{f}\alpha_{f}+\gamma_{s}\alpha_{s} }{\gamma_{f}+\gamma_{s}} \nonumber \\
%	\alpha_{v} &=\frac{\frac{\gamma_{f}}{\gamma_{s}}\alpha_{f}+\alpha_{s}}{ \frac{\gamma_{f}}{\gamma_{s}} + 1}
%\end{align}
%
%\subsubsection*{Estimating variance of $\overline{z}_{t}$}
%
%Taking \eqref{eq:totaldeltazt} we can estimate variance of $\overline{z}_{t}$. Using the same process as \citet{engen_evolution_2011}. Indeed, \eqref{eq:totaldeltazt} has the form:
%\begin{equation}
%	\label{eq:defdeltaat}
%	\Delta A_{t}=-D A_{t} + e_{t}
%\end{equation}
% with $A_{t}=\overline{z_{t}} - \theta_{v}$, $D = G_{I} \gamma$ and $e_{t} = D\alpha_{v}\epsilon_{t}$. \eqref{eq:defdeltaat} has a stationary solution:
%
%\begin{subequations}
%	\begin{equation}
%		\label{at+1}
%		A_{t+1} = (1-D)^{t+1} A_{0} + \sum_{r=0}^{t} (1-D)^{r} e_{t-r}
%	\end{equation}
%	
%	If we consider the evolution of $A_{t}$ over a long time, \eqref{at+1} becomes, because $(1-D) < 1$:
%	
%	\begin{align}
%		\label{atsum}
%		A_{t} = \sum_{r=0}^{\infty}e_{t-r}(1-D)^{r}
%	\end{align}
%\end{subequations}
%
%We want to estimate how will $\overline{z}_{t}$ move away from the mean because of environmental fluctuations, that is why we compute its variance. From \eqref{at+1}:
%\begin{equation}
%	\begin{aligned}
%	\text{Var}(A_{t}) &= \text{Var} \left[ \sum_{r=0}^{\infty}e_{t-r}(1-D)^{r} \right] \\
%	&= \sum_{r=0}^{\infty} \text{Var} \left[ e_{t-r}(1-D)^{r} \right] \\
%	&=  \sum_{r=0}^{\infty} \text{Var} \left[ \epsilon_{t-r} D \alpha_{v} (1-D)^{r} \right] \\
%	&\overset{\text{def}}{=} \sigma_{\epsilon}^{2} D^{2} \alpha_{v}^{2} \sum_{r=0}^{\infty} (1-D)^{2r} \\
%	\text{Var}(A_{t}) &= \sigma_{\epsilon}^{2} D^{2} \alpha_{v}^{2} \frac{1}{1-(1-D)^{2}} \\
%	\text{Var}(\overline{z_{t}}) &\overset{\text{def}}{=} \sigma_{\epsilon}^{2} G_{I}^{2}\gamma^{2} \alpha_{v}^{2} \frac{1}{G_{I}\gamma(2-G_{I}\gamma)} \\
%	\label{eq:varzws}
%	\text{Var}(\overline{z_{t}}) &\overset{\gamma \to 0}{=} \frac{1}{2} G_{I}\gamma\sigma_{\epsilon}^{2}\alpha_{v}^{2}
%	\end{aligned}
%\end{equation}

\label{sec:Res}
\section*{Results}

\subsection*{Constant environment and density-dependence}

We used the previously developed model in \citetext{Sandell et al. 2014, master's thesis} and simulated (see~\autoref{fig:dd}) a tree population for 150 years in a constant environment, with and without density-dependence on $s_0$, assess the effects of a more realistic demography.

As expected, density-dependence allow regulating the population (\autoref{fig:dd} right panel), as the number of mature and immature individuals seem to converge respectively to $18000$ and $10000$ individuals, while without density-dependence the population is exponentially growing.

Looking at the phenotype, we started from exactly the same starting point $z=116$ for phenotypic and genotypic values. Without density-dependence, the population quickly converge to the equilibrium phenotype ($\overline{z_{weak}}$ given by the approximation in~\autoref{eq:zweak}), $\overline{z_{weak}} = 116$ in this case. With density-dependence the equilibrium is shifted towards the survival optimum $\theta_s$ ($\overline{z_{weak, dd}} = 121.8$, $\theta_s = 130$ while $\theta_f = 100$).

The lower seed survival $s_0$ decreases $\gamma_f$~\eqref{eq:gammaf} changing the weights in~\eqref{eq:zweak}, making it more interesting to favor the survival of already established immature trees than the production of many propagules with very little survival prospect.

Within the density-dependent model the mean immature phenotype $\overline{z_I}$ converge quicker than the mean mature phenotype $\overline{z_M}$ to the equilibrium. It is because of stage-structured nature of our model, the mature stage is a combination of individuals that lived for around 40 generations (given our life-cycle), it buffers adaptation. To make $\overline{z_M}$ closer to $\overline{z_{weak}}$, immature individuals with a phenotype closer to $\overline{z_{weak}}$ need to survive long enough to mature and outnumber initial mature individuals with phenotype further from $\overline{z_{weak}}$.

\begin{figure}[ht!]
	\centering
	\includegraphics[scale=1]{Figures/DDphenopop.pdf}
	\caption{\textbf{Effect of density-dependence on phenotypes and populations}. \textbf{Top}: Phenotype variations in population ($\overline{z_I}, \overline{z_M}$, starting from $z = 116$) with their corresponding genotypic values ($\overline{g_I}, \overline{g_M}$), and the approximation given by \autoref{eq:zweak}; \textbf{Bottom}: Population, number of immature individuals ($N_I$, red), number of mature individuals ($N_M$, blue). Starting from Stable-Stage Distribution (SSD) in constant environment. \textbf{No DD} means we used the model without density-dependence, \textbf{DD} means we implemented density-dependence through $s_0$ (see \autoref{eq:ddfunc}).}
	\label{fig:dd}
\end{figure}

\subsection*{Fluctuating optima}

To mimic a more realistic environment we made the optima fluctuate, with various correlations between them. We simulated three populations using the same random seed. We only vary correlations between noises.

Explain in the text correlation of $z_{I}$ with $\theta_{s}(t)$

\begin{figure}[ht!]
	\centering
	\includegraphics[scale=1]{Figures/PhenoLHTwithCorr.pdf}
	\caption{\textbf{Fluctuating optima against constant environment}. \textbf{Top:} comparison of phenotypes from simulations with constant or fluctuating optima, $\overline{z_{eq}}$ is the approximation shown in \autoref{eq:zweak}. \textbf{Bottom:} (\textbf{Left}) life-history traits in constant or fluctuating environment, (\textbf{Right}) population in constant or fluctuating environment, $N_I$ is the number of immature individuals and $N_M$ the number of mature individuals, population started from the stable stage distribution. \textbf{Solid lines:} values in constant environment, \textbf{Dashed lines:} in fluctuating environment.}
	\label{fig:corr}
\end{figure}

\subsection*{Trend in the environment}

\begin{figure}[ht!]
	\centering
	\includegraphics[scale=1]{Figures/Trend.pdf}
	\caption{\textbf{Mixed influences of trend and fluctuations on the population}. \textbf{Top:} Phenotype evolution with and without fluctuations, results from a \textbf{single} simulation; \textbf{Bottom:} (\textbf{Left}) life-history traits evolution; (\textbf{Right}) demography. \textbf{Solid lines:} (\textbf{No fluct.}) linearly decreasing optima with time; \textbf{Dashed lines:} \textbf{With fluct.} fluctuating decreasing optima. Life-history traits and population were averaged over 15 independent population to buffer the stochasticity of simulations.}
	\label{fig:trend}
\end{figure}

We implemented a decreasing trend in $\theta_f$ with fluctuations (\autoref{fig:trend}) to mimic climate change.
We averaged values over 15 simulations for life-history traits while showed a single simulation for the phenotype (respectively bottom and top panels \autoref{fig:trend}).

$s_I$ has an interesting behavior, it first increases, reaches a maximum, then decreases. The decreasing trend in optima variation causes at first the mean population phenotype to move closer to $\theta_s$, thus maximizing $s_I$ values when it crosses $\theta_s$ line, as soon as it moves beyond $s_I$ starts to decrease again. The fluctuations seem to decrease $s_I$ (mean difference of 0.5), it may be a cost associated with the variance of optimum fluctuations, the optimum is often under the mean population value.

On the contrary $f_1$ and $f_2$ do not show a different pattern with or without fluctuations. They decrease because the mean population phenotype go further away from $\theta_f$.

Seed germination and survival $s_0$ is increased by fluctuations, via an indirect mechanism: fluctuations decrease immature survival $s_I$, thus decreasing the immature population $N_I$ and so the mature population $N_M$; this population decrease also decrease competition and density, increasing $s_0$ as it is density-dependent (see~\autoref{eq:ddfunc}).

As expected, the decreasing trend in $\theta_f$ creates a lag between the optima and the mean population values, because adaptation is slower than the rate of change. However, the population can still survive with such a rate if the difference between the optima and the means become constant. On a very long scale (2500 years) it is what happens in this case, the population maintain by changing its phenotype fast enough to track the optima variation (data not shown).

\subsection*{Estimation of the fluctuations}

\begin{figure}[ht!]
	\centering
	\includegraphics[scale=1]{Figures/optsmaps.pdf}
	\caption{\textbf{$\theta_{f}$ estimations from PHENOFIT data}. \textbf{Top:} estimations of $\theta_f$ for each study site (see \nameref{sec:M&M} for details). \textbf{Bottom:} (\textbf{Left}) map of the study sites; (\textbf{Right}) Theoretical fecundity functions with parameters from~\autoref{tab:params} with values of $\theta_f$ equals to $100$, $20$ and $-20$, solid lines indicate achievable phenotype, dashed lines show theoretical curves but unreachable phenotypes.}
	\label{fig:thetaf}
\end{figure}

In 6 localities (map \autoref{fig:thetaf} bottom left) using \textsc{PHENOFIT} output, we computed $\theta_f$ values at these locations (top 3 rows of \autoref{fig:thetaf}). For the 6 sites, predicted $\theta_f$ decrease with time, it is more precocious as time passes. This observation matches the advance of phenology observed in the literature because of climate change.

Over the general trend, we observe a small amplitude variation (with a standard deviation of \SI{9.6}{\day}), corresponding to year to year change in $\theta_f$ and some dramatic decreases in its values, sometimes reaching negative values (For example at BIC site in 1976). The frequency of these events increase with time as they become common after 2050 for all sites. Note that those events are biased towards the decrease of $\theta_f$, as there is no equivalent dramatic increases.

The negative values of $\theta_f$ computed in \autoref{fig:thetaf}, may seem striking as there is no such thing as a negative bud-burst date! It indicates strong directional selection to shorten bud-burst those years with very little sign of quadratic selection on that trait. As bottom right panel of \autoref{fig:thetaf} shows, we can have negative value of $\theta_f$ and still have achievable phenotypes. If $\theta_f$ is very negative for a given year (less than -100 in 2048 for LAB), it means there will be no reproduction this year (flat tail of blue curve, bottom right oanel \autoref{fig:thetaf}).

We excluded those extreme events, taking all sites together, to estimate the trend in the variation of $\theta_f$ (see~\nameref{sec:M&M}). Using linear regression on $\theta_f$ with time, we found a rate of \SI{-0.15}{\day\per\year}, with normal residuals having a variance of \SI{105}{\day\squared} (data not shown, $R^2=0.2341$, $p<2\text{\sc{e}-}16$, $F=186.7$ with $611$ d.f.).

We investigated whether there was a break between years modeled from real data by \textsc{PHENOFIT} (before 2001) and years modeled using climate models with climate change included (from 2001). We performed the same regression as above ($\theta_f$ with time), without taking apart the extreme values, for all sites, splitting the data before 2001 and from 2001 (data no shown). For each site, the slope estimates was higher with extreme events.

\label{sec:Disc}
\section*{Discussion}

We modeled a stage-structured tree population using a quantitative genetics approach, with bud-burst date varying between two optima. We predicted phenotype evolution in the next 150 years. Using \textsc{PHENOFIT} simulation results we computed values for one of the optima. According to simulations, an increasing number of extreme events will happen in the next century where all fecundities will be equal to zero.

As expected in the literature, we found a decreasing trend in the variation of optima, i.e. a trend to precocity of phenology \citep{aitken_adaptation_2008, ehrlen_timing_2009}. Because of the general increase in temperature, organisms advance their phenology to track their original environment, either by genetic change or plasticity~(\citealt{savolainen_genetic_2004}, reviewed in \citealt{merila_climate_2014}). Increasing evidences underline the role of plasticity for trees in adaptation to environmental change, we did not consider it in our model to focus on the specific effects of fluctuations; Phenotypic plasticity should also be taken into account in a future model as it slows down genetic adaption~\citep{alberto_potential_2013, aitken_adaptation_2008}. 

Genotypic ($\overline{g_i}$) and phenotypic ($\overline{z_i}$) values are different in our simulations (see~\autoref{fig:dd}); because of both stage-structure and the two optima we use.

With trend (\autoref{fig:trend}), counter-intuitively, there is no difference of in adaptation speed with and without fluctuations. We could have expected an additional cost of fluctuations: if the population would have tracked fluctuations closely, then a increased noise variance would have decrease fitness dramatically. Since it is not the case for reasons cited above, we see no cost of fluctuations when there is a general trend in the environment.

The mean phenotype in immature individuals $\overline{z_I}$ changes faster than the mean phenotype in mature individuals $\overline{z_M}$, there is a difference of adaptation speed between stages. The stage-structure of our population explains the different behaviors among the two classes. 

Our model was parametrized according to the sessile oak life-cycle (\textit{Quercus petraea} spp.) — using the \textsc{PHENOFIT} simulations to predict fluctuations and trend in the optima — the species does not go extinct in the next 150 years (\autoref{fig:trend}). Climate change has still dramatic demographic consequences, the population halving in less than 100 years. The sessile oak would be left more vulnerable demographically as it increases genetic drift and the potential consequences of dramatic events.

Those estimations and fluctuations do not include the extreme years when all fecundities are equal to zero, i.e. the selection gradient is zero, which should affect adaptation dynamics. Modeling correctly the variations of optima values would lead to more realistic behavior, it could be done by drawing the optima from an almost Gaussian distribution with a very long tail towards negative values.

Quantitative genetics models have been developed to predict populations' adaptation dynamics, studied traits are often modeled as follow: in a given environment there is an optimal trait value, population then converge to this value to maximize fitness if achievable~\citep{lande_quantitative_1982}. Taking into account the different responses to environmental changes in various stages, more stages with also different optima could contribute in the general phenotypic equilibrium.

Those optima are of abstract nature, they are difficult to measure directly in natural populations; to overcome this difficulty, it was suggested to measure selection gradient to have access to fitness landscapes and extrapolate optima~\citep{lynch_evolution_1993}. In the wild, a whole population do not share the same bud-burst date, there exists a range of possible dates that all maximize fitness in the same way (any good refs on this?). Including such optimal ranges could be one step towards more realistic quantitative genetics models.

%TC:ignore
%\label{sec:Ack}
\section*{Authors Contributions and Acknowledgments}

M. Grenié did the analyses and simulations, based on L. Sandell's work. O. Ronce and LM. Chevin supervised the project. A. Duputié shared \textsc{PHENOFIT} outputs. I. Chuine advised the project about \textsc{PHENOFIT} outputs.

I would like to thank Ophélie Ronce for being an extremely patient supervisor, Luis-Miguel Chevin and Isabelle Chuine for the discussions we had about the project. I thank Guillaume Martin for being an awesome office-mate. Finally, I would like to thank the whole "Metapopulations" team at ISEM for their welcome.
%TC:endignore

\bibliographystyle{cell.bst}
\bibliography{Bibliography}
\bibliographystyle{cell.bst}
\bibliography{Bibliography}

\begin{table}
\begin{center}
	\rowcolors{1}{white}{lightgray}
	\begin{tabular}{l c c}
		\hline \hline
		Parameter & Notation & Value \\
		\hline
		\multicolumn{3}{l}{\textbf{Life Cycle}} \\
		Optimal phenotype for fecundity & $\theta_{f}$ & 100 \\
		Optimal phenotype for immature survival & $\theta_{s}$ & 130 \\
		Fecundity function width & $\omega_{f}$ & 400 \\
		Survival function width & $\omega_{s}$ & 400 \\
		Heritability & $h^2$ & 0.5 \\
		Phenotypic variance of immatures & $P_{I}$ & 40 \\
		Phenotypic variance of matures & $P_{M}$ & 40 \\
		Genotypic variance of immatures & $G_{I} = P_{I} \times h^2$ & 20 \\
		Genotypic variance of matures & $G_{M}$ & 20 \\
		Survival of immature at phenotypic optimum & $\overline{s_{I}}(\overline{z} = \theta_{s})$ & 0.8 \\
		Fecundity of first time reproducers at optimum & $\overline{f_{1}}(\overline{z} = \theta_{f})$ & 100 \\
		Fecundity of experienced reproducers at optimum & $\overline{f_{2}}(\overline{z} = \theta_{f})$ & 200 \\
		Maturation rate of immature & $m$ & 0.02 \\
		Combined survival and germination rate of seed & $s_{0}$ & 0.03 \\
		Survival of mature stage & $s_{M}$ & 0.99 \\
		\multicolumn{3}{l}{\textbf{Density-dependence}} \\
		Maximum $s_{0}$ in density-dependence function & $s_{0, max}$ & 0.12 \\
		Decreasing factor due to immatures & $k_{I}$ & 0.001 \\
		Decreasing factor due to matures & $k_{M}$ & 0.005 \\
		\multicolumn{3}{l}{\textbf{Fluctuations}} \\
		Sensitivity of optimum for fecundity to fluctuation & $\alpha_{f}$ & 5 \\
		Sensitivity of optimum for survival to fluctuation & $\alpha_{s}$ & 5 \\
		Noise variance for fecundity & $\sigma_{\xi_{f}}^2$ & 3.725 \\
		Noise variance for survival & $\sigma_{\xi_{s}}^2$ & 3.725 \\
		Correlation between noises & $\rho_{N}$ & 0.5 \\
		Trend coefficient & $k$ & -0.15 \\
		\hline \hline
	\end{tabular}
	\caption{Standard parameter set}
	\label{tab:params}
\end{center}
\end{table}

\begin{figure}[ht!]
	\centering
	\includegraphics[scale=1]{Figures/DDphenopop.pdf}
	\caption{\textbf{Effect of density-dependence on phenotypes and populations}. \textbf{Top}: Phenotype variations in population ($\overline{z_I}, \overline{z_M}$, starting from $z = 116$) with their corresponding genotypic values ($\overline{g_I}, \overline{g_M}$), and the approximation given by \autoref{eq:zweak}; \textbf{Bottom}: Population, number of immature individuals ($N_I$, red), number of mature individuals ($N_M$, blue). Starting from Stable-Stage Distribution (SSD) in constant environment. \textbf{No DD} means we used the model without density-dependence, \textbf{DD} means we implemented density-dependence through $s_0$ (see \autoref{eq:ddfunc}). Here, bud-burst date is expressed in julian days (numbered days in the year, 1st of January being 1 in julian days).}
	\label{fig:dd}
\end{figure}

\begin{figure}[ht!]
	\centering
	\includegraphics[scale=1]{Figures/PhenoLHTwithCorr.pdf}
	\caption{\textbf{Fluctuating optima against constant environment}. \textbf{Top:} comparison of phenotypes from simulations with constant or fluctuating optima, $\overline{z_{eq}}$ is the approximation shown in \autoref{eq:zweak}. \textbf{Bottom:} (\textbf{Left}) life-history traits in constant or fluctuating environment, (\textbf{Right}) population in constant or fluctuating environment, $N_I$ is the number of immature individuals and $N_M$ the number of mature individuals, population started from the stable stage distribution. \textbf{Solid lines:} values in constant environment, \textbf{Dashed lines:} in fluctuating environment.}
	\label{fig:corr}
\end{figure}

\begin{figure}[ht!]
	\centering
	\includegraphics[scale=1]{Figures/Trend.pdf}
	\caption{\textbf{Mixed influences of trend and fluctuations on the population}. \textbf{Top:} Phenotype evolution with and without fluctuations, results from a \textbf{single} simulation; \textbf{Bottom:} (\textbf{Left}) life-history traits evolution; (\textbf{Right}) demography. \textbf{Solid lines:} (\textbf{No fluct.}) linearly decreasing optima with time; \textbf{Dashed lines:} \textbf{With fluct.} fluctuating decreasing optima. Life-history traits and population were averaged over 15 independent population to buffer the stochasticity of simulations.}
	\label{fig:trend}
\end{figure}

\begin{figure}[ht!]
	\centering
	\includegraphics[scale=1]{Figures/optsmaps.pdf}
	\caption{\textbf{$\theta_{f}$ estimations from PHENOFIT data}. \textbf{Top:} estimations of $\theta_f$ for each study site (see \nameref{sec:M&M} for details). \textbf{Bottom:} (\textbf{Left}) map of the study sites; (\textbf{Right}) Theoretical fecundity functions with parameters from~\autoref{tab:params} with values of $\theta_f$ equals to $100$, $20$ and $-20$, solid lines indicate achievable phenotype, dashed lines show theoretical curves but unreachable phenotypes.}
	\label{fig:thetaf}
\end{figure}

\end{document}