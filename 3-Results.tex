\label{sec:Res}
\section*{Results}

\subsection*{Constant environment and density-dependence}

We used the previously developed model in \citetext{Sandell 2014} and simulated (see~\autoref{fig:dd}) a tree population for 150 years in a constant environment, with and without density-dependence on $s_0$, assess the effects of a more realistic demography.

As expected, density-dependence allows regulating the population (\autoref{fig:dd} right panel), as the number of mature and immature individuals seem to converge respectively to $18000$ and $10000$ individuals, while without density-dependence the population is exponentially growing.

Looking at the phenotype, we started from exactly the same starting point $z=116$ for phenotypic and genotypic values. Without density-dependence, the population quickly converge to the equilibrium phenotype ($\overline{z_{weak}}$ given by the approximation in~\autoref{eq:zweak}), $\overline{z_{eq}} = 116$ in this case. With density-dependence the equilibrium phenotype is shifted towards the survival optimum $\theta_s$ ($\overline{z_{weak, dd}} = 121.8$, $\theta_s = 130$ while $\theta_f = 100$). The lower seed survival $s_0$ decreases $\gamma_f$~\eqref{eq:gammaf} changing the weights in~\eqref{eq:zweak}, making it more interesting to favor the survival of already established immature trees than the production of many propagules with very little survival prospect.

The genotypic values, $\overline{g_I}$ and $\overline{g_M}$ (respectively gray and black on \autoref{fig:dd}) are clearly distinct from the mean phenotype values.

The mean immature phenotype $\overline{z_I}$ converge quicker than the mean mature phenotype $\overline{z_M}$ to $\overline{z_{eq}}$. High mature trees survival in our simulations makes it long to replace them with a different genotype and phenotype. To make $\overline{z_M}$ closer to $\overline{z_{eq}}$, immature individuals with a phenotype closer to $\overline{z_{eq}}$ need to survive long enough to mature and outnumber initial mature individuals with phenotype further from $\overline{z_{eq}}$.

\subsection*{Fluctuating optima}

To mimic a changing environment we made the optima fluctuate (\autoref{fig:corr}, dashed lines) and compared this model to the one in constant environment (solid lines).

The mean phenotype of the population does not change very much with the fluctuations, indeed, $\overline{z_M}$ in constant and fluctuating environment are equal, and they are also equal to $\overline{z_{eq}}$, that is why they are indistinguishable on \autoref{fig:corr}.

Only $\overline{z_I}$ fluctuates under varying environment, but the fluctuations have a very small variance compared to the ones of the optima. We found a strong correlation between $\theta_s$ and $\overline{z_I}$ across years ($\rho_{\text{Pearson}} = 0.6997364$) than with $\theta_f$ It shows how immature individuals track the variations of the survival optimum and invest specifically more on this fitness component.

Because of the variations the phenotype lags away from the optimal value,decreasing the mean of $s_I$ in fluctuating environment. The number of immature individuals $N_I$ is thus lower under the fluctuating regime then decreasing the number of mature individuals $N_M$, this decline in density in turn increases $s_0$. The variation in $\theta_s$ causes $s_I$ to decrease, it reveals the cost of the fluctuations demographically: fluctuating regime causes variations in survival that may have dramatic effect on population.

However, those fluctuations do not seem to affect fecundities $f_1$ and $f_2$ in the same way (\autoref{fig:corr} bottom left panel). As the mean optima move they get closer to population phenotype increasing fecundity, but the next time step they move further away from this phenotype and they decrease fecundity.

The asymmetry of responses between  survival $s_I$ and  fecundities $f_1$ and $f_2$, are explained by the specific trade-off occurring in our population. The mean phenotype in our simulations is closer to $\theta_s$ than to $\theta_f$, there is a higher chance of $\theta_s$ to be lower or much higher than the mean population phenotype, then there is for $\theta_f$ to cross the mean population phenotype line.

As we had partially correlated noises in our population (see \autoref{tab:params} to have standard parameters set), we vary correlations for noises between 0 and 1. The results where similar whatever the correlation coefficient. It seemed that the lower the correlation between noises, the higher were the demographic burden (results unshown). Uncorrelated environments decrease more the life-history traits than correlated environments.

\subsection*{Trend in the environment}

We implemented a decreasing trend in $\theta_f$ with fluctuations (\autoref{fig:trend}) to mimic climate change. We simulated both a linear trend and a linear trend with fluctuations in optima variation (respectively solid and dashed lines in \autoref{fig:trend}).

The phenotype in the population decreases as the optima decrease, but much slower, whether with fluctuations or not. The mean phenotype in the immature stage $\overline{z_I}$ seem to vary in the same fashion with and without fluctuations, while the mean phenotypes among mature individuals $\overline{z_M}$ are almost indistinguishable in the two types of environments, $\overline{z_M}$ under fluctuations (dashed line) is a little bit over $\overline{z_M}$ without them (solid line). We implemented the approximation $\overline{z_\epsilon}$ from \citep{engen_evolution_2011}, it seems to follow the variations in a similar fashion as the mean phenotype of the mature individuals.

The immature survival ($s_I$) has an interesting behavior, it first increases, reaches a maximum, then decreases. The decreasing trend in optima variation causes at first the mean population phenotype to move closer to $\theta_s$, thus maximizing $s_I$ values when it crosses $\theta_s$ line, as soon as it moves beyond $s_I$ starts to decrease again.

On the contrary $f_1$ and $f_2$ the mean population phenotype go further away from $\theta_f$ with and without fluctuation.

After a certain number of years, population is lower in environments with trends than in constant ones, the effect is especially visible in the environments without fluctuations (\autoref{fig:corr} \& \autoref{fig:trend} bottom right panel). However, fluctuations in constant environment seem to decrease more the number of immature individuals than the trend does in constant environment. 

As expected, the decreasing trend in $\theta_f$ creates a lag between the optima and the mean population values, because adaptation is slower than the rate of change. However, the population can still survive with such a rate if the difference between the optima and the means become constant. On a very long scale (2500 years) it is what happens in this case, the population maintains itself by changing its phenotype fast enough to track the optima variation (data not shown).

\subsection*{Estimation of the fluctuations}

In 6 localities (map \autoref{fig:thetaf} bottom left) using \textsc{PHENOFIT} output, we computed $\theta_f$ values at these locations (top 3 rows of \autoref{fig:thetaf}). For the 6 sites, predicted $\theta_f$ decreases with time: earlier bud-burst is favored as climate warms.

Over the general trend, we observe a small amplitude variation, corresponding to year to year change in $\theta_f$ and some more episodic dramatic decreases in its values, sometimes reaching negative values (e.g. at BIC site in 1976). The frequency of these events increase with time as they become common after 2050 for all sites. Note that those events are biased towards the decrease of $\theta_f$, as there is no equivalent dramatic increases.

The negative values of $\theta_f$ computed in \autoref{fig:thetaf}, may seem striking as there is no such thing as a negative bud-burst date! It indicates strong directional selection to shorten bud-burst those years with very little sign of quadratic selection on that trait. As bottom right panel of \autoref{fig:thetaf} shows, we can have negative value of $\theta_f$ and still have achievable phenotypes. If $\theta_f$ is very negative for a given year (less than -100 in 2048 for LAB), it means there will be no reproduction this year (flat tail of blue curve, bottom right panel \autoref{fig:thetaf}).

We excluded those extreme events, taking all sites together, to estimate the trend in the variation of $\theta_f$ (see~\nameref{sec:M&M}). Using linear regression on $\theta_f$ with time, we found a rate of $-0.15 \text{day}.\text{year}^{-1}$, with normal residuals having a variance of $105 \text{day}^2$ (data not shown, $R^2=0.2341$, $p<2\text{\sc{e}-}16$, $F=186.7$ with $611$ d.f.).

%We investigated whether there was a break between years modeled from real data by \textsc{PHENOFIT} (before 2001) and years modeled using climate models with climate change included (from 2001). We performed the same regression as above ($\theta_f$ with time), without taking apart the extreme values, for all sites, splitting the data before 2001 and after 2001 (data not shown). For each site, the slope estimates were non-significantly different from zero for years before 2001, while we had very steep slopes estimates after 2001 between \SI{-0.7}{\day\per\year} and \SI{-1.1}{\day\per\year}.