\label{sec:Disc}
\section*{Discussion}

We modeled a stage-structured tree population using a quantitative genetics approach, with bud-burst date varying between two optima. We predicted phenotype evolution in the next 150 years. Using \textsc{PHENOFIT} simulation results we computed values for one of the optima. According to simulations, an increasing number of extreme events will happen in the next century where all fecundities will be equal to zero.

As expected in the literature, we found a decreasing trend in the variation of optima, i.e. a trend to precocity of phenology \citep{aitken_adaptation_2008, ehrlen_timing_2009}. Because of the general increase in temperature, organisms advance their phenology to track their original environment, either by genetic change or plasticity~(\citealt{savolainen_genetic_2004}, reviewed in \citealt{merila_climate_2014}). Increasing evidence underlines the role of plasticity for trees in adaptation to environmental change, we did not consider it in our model to focus on the specific effects of fluctuations on genetic adaptation as a first step in our exploration; phenotypic plasticity should also be taken into account in a future model as it slows down genetic adaption~\citep{alberto_potential_2013, aitken_adaptation_2008}. 

With trend (\autoref{fig:trend}), counter-intuitively, there is no difference of in adaptation speed with and without fluctuations. We could have expected a cost of fluctuations for adaptation: if the population would have tracked fluctuations closely, noise fluctuations would have moved away optima from mean population phenotype for sure, decreasing fitness. As a long-lived and stage-structured species, the oak population in our model do not follow very closely those fluctuations, that is why we see no cost of fluctuations.

The mean phenotype in immature individuals $\overline{z_I}$ changes faster than the mean phenotype in mature individuals $\overline{z_M}$, there is a difference of adaptation speed between stages. The stage-structure of our population explains the different behaviors among the two classes. 

Our model was parametrized according to the sessile oak life-cycle (\textit{Quercus petraea} spp.) — using the \textsc{PHENOFIT} simulations to predict fluctuations and trend in the optima. The species does not go extinct in the next 150 years (\autoref{fig:trend}). Climate change has still dramatic demographic consequences, the population halving in less than 100 years. The sessile oak would be left more vulnerable demographically as it increases genetic drift and the potential consequences of dramatic events.

Those estimations and fluctuations did not included the extreme years when all fecundities were equal to zero, i.e. the selection gradient is zero, nor those where the optima was very negative, with very strong directional selection, those years should affect adaptation dynamics. Modeling correctly the variations of optima values would lead to more realistic behavior, it could be done by drawing the optima from an almost Gaussian distribution with a very long tail towards negative values.

Taking into account the different responses to environmental changes in various stages, more than 2 stages with their own optima could contribute in the general phenotypic equilibrium.

Those optima are of abstract nature, they are difficult to measure directly in natural populations; to overcome this difficulty, it was suggested to measure selection gradient to have access to fitness landscapes and extrapolate optima~\citep{lynch_evolution_1993}. In the wild, a whole population do not share one single bud-burst date, they do not forcefully converge to a single optimal date; rather there exists a range of possible dates that all maximize fitness in the same way. Each individual bud-burst date may not be optimal but they converge if achievable in a range of different optimal values. Including such optimal ranges could be one step towards more realistic quantitative genetics models.