\label{sec:Disc}
\section*{Discussion}

We modeled a stage-structured tree population using a quantitative genetics approach, with bud-burst date the studied trait varying between two optima. We used it to understand their phenotype evolution in the next 150 years. Using \textsc{PHENOFIT} simulation results we computed values for one of the optima. An increasing number of extreme events would happen in the next century.

As expected in the literature, we found a decreasing trend in the variation of optima, i.e. a trend to precocity of phenology \citep{aitken_adaptation_2008, ehrlen_timing_2009}. Because of the general increase in temperature, organisms advance their phenology to track their original environment.

Long-lived stage structure populations don't have show strong reaction to fluctuations nor to climate change -> beware of the "buffer" effect, due to life cycle

Fluctuations don't have strong effect on phenotype but on demography 

Increasing number of catastrophic events

Our model -> need more realistic variations including "catastrophic events", include phenotypic plasticity, understand better the difference between breeding values and phenotypes, predict extinction time?