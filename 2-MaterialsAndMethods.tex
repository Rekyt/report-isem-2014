\section*{Materials and Methods}
\label{sec:M&M}

\subsection*{Population model}

We used a previously developed model with stage-structure \citetext{Sandell 2014, master's thesis}. We considered a population of trees split in two classes, immature (I) and mature (M). only mature individuals reproduce. Each year, an immature individual can survive with a probability $s_I$, mature and reproduce with a probability of $m$. At the same time, a mature individual has a probability $s_M$ to survive. First-time reproducers, i.e. immature that became mature and reproduce the same year, have a fecundity of $f_1$, while experienced reproducers, those who already reproduced at least once, have a fecundity of $f_2$. Produced seeds have a probability $s_0$ to survive and join the pool of immature trees. The standard parameters set is given in (\autoref{tab:params}). The population census is just before reproduction. The population dynamics can be predicted using the following matrix \citep{caswell_matrix_2001}:
\begin{equation}
	\label{eq:popmat}
	A =
	\begin{pmatrix}
	a_{II} & a_{IM} \\
	a_{MI} & a_{MM}
	\end{pmatrix}
	=
	\begin{pmatrix}
	s_{0} m f_{1} + s_{I} (1 - m) & s_{0} f_{2} \\
	s_{M} m & s_{M}
	\end{pmatrix}
\end{equation}

, where $a_{ij}$, the transition rate, describes the contribution of stage $j$ individuals to stage $i$ the next year. With given initial conditions we can compute the number of individuals in the two stages by iterating matrix multiplication by $A$.

We implemented density-dependence in this population, so that the population would not continuously increase (see \autoref{fig:dd}). We assumed seed germination and survival parameter $s_{0}$ declined with increasing density of mature and immature competitors using a Beverton-Holt function to avoid chaotic behaviors \citep{caswell_matrix_2001}:
\begin{equation}
	\label{eq:ddfunc}
	s_{0} = \frac{s_{0, max}}{1 + k_{I} N_{I} + k_{M} N_{M}}
\end{equation}

with $k_{I}$ and $k_{M}$ the weights of immature ($N_{I}$) and mature ($N_{M}$) population respectively. $s_{0, max}$ is the maximum achievable $s_{0}$.

\subsection*{Phenotype and life-history traits}

In this population we observed a single phenotype $z$: the bud-burst date. Here, bud-burst date is expressed in julian days (numbered days in the year, 1st of January being 1 in julian days). In our model, an individual is born with a given phenotype and keeps it throughout his life.

We supposed certain life-history traits for each individual - $s_{I}$ immature survival, $f_1$ first reproducers fecundity and $f_2$ experienced reproducers - (see \autoref{eq:popmat}) to be Gaussian function of phenotype $z$. Thus, bud-burst date directly influence their values. They can are expressed as follow:

\begin{equation}
	\label{eq:indlht}
	\left\{
	\begin{aligned}
	s_{I}(z) &= s_{I}(\theta_{s})	\exp\left(-\frac{(z - \theta_{s})^2}{2\omega_{s}}\right) \\
	f_1(z) &= f_1(\theta_{f})	\exp\left(-\frac{(z - \theta_{f})^2}{2\omega_{f}}\right) \\
	f_2(z) &= f_2(\theta_{f})	\exp\left(-\frac{(z - \theta_{f})^2}{2\omega_{f}}\right)
	\end{aligned}
	\right.
\end{equation}


, $\theta_s$ is the optimal bud-burst date for survival, i.e. phenotype where $s_I$ is at its maximum $s_I(\theta_s)$; $\omega_s$ is the width of the Gaussian function, its inversely related to selection intensity: with small $\omega_s$ values, only a restricted range of bud-burst dates would have important survival rates. $f_1$ and $f_2$ have similar expressions, but the optimal bud-burst date $\theta_f$ is different from $\theta_s$, $f_1$ and $f_2$ only differ by their maximum values $f_i(\theta_f)$, with $f_1$ lower than $f_2$ (see \autoref{tab:params} to have standard parameters values).

The optimal trait values $\theta_s$ and $\theta_f$ differ between stages and life-history components, but trait value does not change along the life of an individual, then there is a trade-off between the two fitness components. And the evolution of the trait affects the life-history of the individual.

If we want to compute mean transition rate $\overline{a_{ij}}$, we need to average $s_I$, $f_1$ and $f_2$ (ex: $\overline{a_{IM}} = s_0 \overline{f_2}$):

\begin{equation}
	\mathrm{E}[s_I] = \overline{s_I} = \int p_I(z)s_I(z)\,\text{d}z
\end{equation}
, with $p_I(z)$ the distribution of $z$ in the immature stage, as we study a quantitative trait we suppose $p_I$ has a Gaussian distribution with mean $\overline{z_I}$ and width $P_I$ the phenotypic variance in the immature stage~\citep{lande_quantitative_1982}. We end with the following expression for $\overline{s_I}$:

\begin{equation}
	\label{eq:poplht}
	\overline{s_{I}}(\overline{z_{I}}) = s_{I}(\theta_{s}) \sqrt{\frac{\omega_{s}}{\omega_{s}+P_{I}}}	\exp\left(-\frac{(\overline{z_{I}} - \theta_{s})^2}{2(\omega_{s}+P_{I})}\right)
\end{equation}

We obtain similar expressions for $\overline{f_1}$ and $\overline{f_2}$.	

\subsection*{Iterations at each time step}

Assuming the phenotype has a Gaussian distribution,  the mean genotypic value of matures and immatures at the next time step is given by (\citealt{barfield_evolution_2011} Eq.5) :

\begin{subequations}
	\begin{align}
		\label{eq:genotypic}
		\overline{g_{I}}' &= (c_{I M} \overline{g_{M}} + c_{I I} \overline{g_{I}}) + 
			(c_{I M} G_M \beta_{a_{IM}} + c_{I I} G_I \beta_{a_{II}}) \\
		\overline{g_{M}}' &=	 (c_{M I} \overline{g_{I}} + c_{M M} \overline{g_{M}})  +
				(c_{M I} G_I \beta_{a_{MI}} + c_{M M} G_M \beta_{a_{MM}})
	\end{align}
\end{subequations}

with $c_{ij} = \frac{n_j \overline{a_{ij}}}{n_i'}$, the contribution of stage $j$ individuals to next years pool of stage $i$ individuals, as a fraction of $i$ individuals at the next time step $n_i'$; and $\beta_{a_{ij}}$ the selection gradient as $\beta_{a_{ij}} = \frac{\partial \ln \overline{a_{ij}}}{\partial \overline{z}}$~\citep{barfield_evolution_2011}. The selection gradient represent the force of directional selection~\citep{lande_quantitative_1982}.

The first term is a weighted average of mean genotypes contributing to this stage; while the second shows the effect of selection.

To have the formal expressions of $\beta_{a_{ij}}$ we need to compute the selection gradients on life-history components:

\begin{equation}
	\label{eq:selgradlfh}
	\begin{aligned}
	\beta_{\overline{s_I}} &= \frac{\partial \ln \overline{s_I}}{\partial \overline{z_I}} &= \frac{\theta_s - \overline{z_I}}{\omega_s + P_I} \\
	\beta_{\overline{f_1}} &= \frac{\partial \ln \overline{f_1}}{\partial \overline{z_I}} &= \frac{\theta_f - \overline{z_I}}{\omega_f + P_I} \\
	\beta_{\overline{f_2}} &= \frac{\partial \ln \overline{f_2}}{\partial \overline{z_M}} &= \frac{\theta_f - \overline{z_M}}{\omega_f + P_M}
	\end{aligned}
\end{equation}

And because we have for example $\overline{a_{II}} = s_0 m \overline{f_1} + \overline{s_I} (1-m)$ we get the selection gradient:

\begin{equation}
	\label{eq:selgradaII}
	\beta_{a_{II}} = \frac{s_0 m \overline{f_1} \beta_{\overline{f_1}} + \overline{s_I} \beta_{\overline{s_I}} (1-m)}{\overline{a_{II}}}
\end{equation}

We have a similar recursion for phenotypes \citep{barfield_evolution_2011}. They depend on terms of direct transition of individuals from one stage to the other $\overline{t_{ij}}$ and events leadings to new individuals $\overline{f_{ij}}$ (and we have $\overline{a_{ij}} = \overline{t_{ij}} + \overline{f_{ij}}$):

\begin{subequations}
	\begin{align}
	\label{eq:phenotypic}
	\overline{z_I'} &= c_{II}^t (\overline{z_I} + P_I \beta_{t_{II}}) + c_{II}^f (\overline{g_I} + G_I \beta_{f_{II}}) + c_{IM}^f (\overline{g_M} + G_M \beta_{f_{IM}}) \\
	\overline{z_M'} &= c_{MI}^t (\overline{z_I} + P_I + \beta_{t_{MI}}) + c_{MM}^t (\overline{z_M} + P_M + \beta_{t_{MM}})
	\end{align}
\end{subequations}

, with $\beta_{t_{II}}$ the gradient of selection defined as above in \autoref{eq:genotypic}, i.e. $\beta_{t_{II}} = \frac{\partial \ln \overline{t_{II}}}{\partial \overline{z_I}}$; $c_{ij}^t = \frac{n_j \overline{t_{ij}}}{n_i'}$ the contribution by direct transition of stage $j$ to stage $i$ and $c_{ij}^f = \frac{n_j \overline{f_{ij}}}{n_i'}$ the contribution by birth.

\subsection*{Approximation under weak selection}

Under weak selection, the mean phenotype at equilibrium in the population $\overline{z}$ follows in constant environment \citep{engen_evolution_2011}:

\begin{equation}
	\label{eq:zweak}
	\overline{z_{eq}} = \frac{\gamma_{f}\theta_{f} + \gamma_{s}\theta_{s}}{\gamma_{f} + \gamma_{s}}
\end{equation}
, with,
\begin{subequations}
	\begin{equation}
	\label{eq:gammaf}
	\gamma_{f} = \frac{v_{I} u_{I} s_{0} m \overline{f_{1}} }{\lambda(P_{I}+\omega_{f})} + \frac{ v_{I} u_{M} \frac{G_{M}}{G_{I}} s_{0} \overline{f_{2}}}{\lambda ( P_{M} + \omega_{f} )}
	\end{equation}
	and
	\begin{equation}
	\label{eq:gammas}
	\gamma_{s} = \frac{ v_{I} u_{I} \overline{s_{I}} (1-m) }{\lambda(P_{I}+\omega_{s})}
	\end{equation}
\end{subequations}

$\gamma_f$ and $\gamma_s$ represent the respective weight of each of the optimum in the trade-off between $\theta_f$ and $\theta_s$ for $\overline{z_{eq}}$. Indeed, if $\theta_f = \theta_s$ then $\overline{z_{eq}} = \theta_f = \theta_s$. But if $\theta_f \neq \theta_s$, then the mean phenotype on the trade-off depends on $\gamma_f$ and $\gamma_s$ and the ratio between them.

\subsection*{Fluctuating optimums}

During my internship, I tried to mimic environmental fluctuations, by making the optima fluctuate as such:
\begin{equation}
\left\{
	\begin{aligned}
		\theta_f(t) &= \overline{\theta_f} + \alpha_f \xi_f \\
		\theta_s(t) &= \overline{\theta_s} + \alpha_s \xi_s
	\end{aligned}
\right.
\end{equation}

$\alpha_i$ is the sensitivity of $\theta_i$ to noise $\xi_i$. $\xi_f$ and $\xi_s$ are noise vectors drawn at each time step from a bi-variate normal distribution with respectively $\sigma_f^2$ and $\sigma_s^2$ variances and correlation $\rho_N$. Thus we get normal fluctuations, correlated with a correlation coefficient of $\rho_N$.

Under varying environment we get an another approximation under weak selection from \citep{engen_evolution_2011} describing the change of mean phenotype:

\begin{equation}
	\label{eq:zfluct}
	\Delta \overline{z}(t) = - G_I \gamma (\overline{z}(t) - \theta_v(t))
\end{equation}
, with
\begin{subequations}
	\begin{align}
		\gamma &= \gamma_f + \gamma_s \\
		\theta_v(t) &= \overline{z_{eq}} + \xi_v \\
		\xi_v &= \frac{\alpha_f \xi_f + \alpha_s \xi_s}{\alpha_f + \alpha_s}
	\end{align}
\end{subequations}

We see that the change in the mean phenotype depends on the sensitivity of the optima $\alpha_i$ as well as on the magnitude of the variations.

\subsection*{Trend in change}

To model climate-change, and especially the trend to increase temperature with time, we included a trend in the variation of the optima. The optima still experience fluctuations as above they linearly vary with time:

\begin{equation}
	\label{eq:kt}
	\left\{
	\begin{aligned}
		\theta_i(t) &= \overline{\theta_i} + \alpha_i\epsilon(t) \\
		\epsilon(t) &= kt + \xi_i
	\end{aligned}
	\right.
\end{equation}

With $k$ having a negative value, the optima decrease with time.

\subsection*{\textsc{PHENOFIT} data}

\textsc{PHENOFIT} is a phenology model including several sub-models, from environmental and phenological data it simulates the survival and reproduction of an average tree to predict its range~\citep{morin_tree_2008}.

 We used output from \textsc{PHENOFIT} (simulations performed by A. Duputié) from 1950 to 2100 for the sessile oak (\textit{Quercus petraea}) about predicted bud burst date and predicted fitnesses in 6 localities (see \autoref{fig:thetaf}). We had fitness predictions for phenotype around the modeled date (a range of 21 days). From these data we predicted the optima fluctuations. Considering fecundity $f$ as a Gaussian function around this date with the same form as $f_1$ in \autoref{eq:poplht}:

\begin{equation}
	\label{eq:beta1}
	\beta = \frac{\partial \ln f}{\partial \overline{z}} = \frac{\theta_f - \overline{z}}{\omega_f + \sigma_z^2}
\end{equation}

Using \citep{lande_measurement_1983}, with $z$ Gaussian, $p(z)$ the distribution of $z$ in the population, $f(z)$ the fitness associated with $z$ and $\overline{f}$ the mean fitness in the population, we computed selection gradients from PHENOFIT simulation outputs as::

\begin{equation}
	\label{eq:beta2}
	\beta = \frac{\text{cov}(z, \frac{f(z)}{\overline{z}})}{\sigma_z^2}
\end{equation}

From \eqref{eq:beta1} and \eqref{eq:beta2} we can express $\theta_f$:

\begin{equation}
	\theta_f = \frac{\text{cov}(z, \frac{f(z)}{\overline{z}})}{\sigma_z^2} (\omega_f + \sigma_z^2) + \overline{z}
\end{equation}

In our estimations we considered $p(z)$ to be Gaussian around the modeled date by \textsc{PHENOFIT}, with a variance of $P_I=40$ as in our analytic model. We normalized this distribution so that all dates in the population would be in the 42 days interval around the modeled date.

\subsection*{Trend analyses}

All statistical analyses were made using R~\citep{R_2014}, graphics were drawn using ggplot2~\citep{ggplot2_2009}, data were handled using dplyr~\citep{dplyr_2014}.

To estimate the trend of the $\theta_f$ variations, we considered a trend model with three components: a general decreasing linear trend, a white noise component with a constant variance and a more dramatic noise leading to "catastrophic" events, with negative $\theta_f$ values.

The regular noise and the trend were estimated excluding those catastrophic events, we kept only value of $\theta_f$ over 60, which is the lower bound of the realizable range of bud burst date of oak trees. Then we performed a linear regression between values of $\theta_f$ and time, giving us an estimation of $k$ from \autoref{eq:kt}. Analyzing the residuals gives us the variance of $\alpha_f \xi_f$ from the same equation.

\begin{table}
\begin{center}
	\rowcolors{1}{white}{lightgray}
	\begin{tabular}{l c c}
		\hline \hline
		Parameter & Notation & Value \\
		\hline
		\multicolumn{3}{l}{\textbf{Life Cycle}} \\
		Optimal phenotype for fecundity & $\theta_{f}$ & 100 \\
		Optimal phenotype for immature survival & $\theta_{s}$ & 130 \\
		Fecundity function width & $\omega_{f}$ & 400 \\
		Survival function width & $\omega_{s}$ & 400 \\
		Heritability & $h^2$ & 0.5 \\
		Phenotypic variance of immatures & $P_{I}$ & 40 \\
		Phenotypic variance of matures & $P_{M}$ & 40 \\
		Genotypic variance of immatures & $G_{I} = P_{I} \times h^2$ & 20 \\
		Genotypic variance of matures & $G_{M}$ & 20 \\
		Survival of immature at phenotypic optimum & $\overline{s_{I}}(\overline{z} = \theta_{s})$ & 0.8 \\
		Fecundity of first time reproducers at optimum & $\overline{f_{1}}(\overline{z} = \theta_{f})$ & 100 \\
		Fecundity of experienced reproducers at optimum & $\overline{f_{2}}(\overline{z} = \theta_{f})$ & 200 \\
		Maturation rate of immature & $m$ & 0.02 \\
		Combined survival and germination rate of seed & $s_{0}$ & 0.03 \\
		Survival of mature stage & $s_{M}$ & 0.99 \\
		\multicolumn{3}{l}{\textbf{Density-dependence}} \\
		Maximum $s_{0}$ in density-dependence function & $s_{0, max}$ & 0.12 \\
		Decreasing factor due to immatures & $k_{I}$ & 0.001 \\
		Decreasing factor due to matures & $k_{M}$ & 0.005 \\
		\multicolumn{3}{l}{\textbf{Fluctuations}} \\
		Sensitivity of optimum for fecundity to fluctuation & $\alpha_{f}$ & 5 \\
		Sensitivity of optimum for survival to fluctuation & $\alpha_{s}$ & 5 \\
		Noise variance for fecundity & $\sigma_{\xi_{f}}^2$ & 3.725 \\
		Noise variance for survival & $\sigma_{\xi_{s}}^2$ & 3.725 \\
		Correlation between noises & $\rho_{N}$ & 0.5 \\
		Trend coefficient & $k$ & -0.15 \\
		\hline \hline
	\end{tabular}
	\caption{Standard parameter set}
	\label{tab:params}
\end{center}
\end{table}