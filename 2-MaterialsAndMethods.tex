\section*{Materials and Methods}

Basic explanation of the models. We modeled a stage-structured population in two stages: immatures and matures. The demography is given by a transition matrix, with...

\subsection*{Under constant environment}

Using \citet{lande_adaptation_2009}, under weak selection we have:
\begin{equation}
	\label{eq:dz}
	\Delta\overline{z} = \frac{d\ln\overline{\lambda}(\overline{z})}{d\overline{z}} = \frac{1}{\overline{\lambda}(\overline{z})} \frac{d\overline{\lambda}(\overline{z})}{d\overline{z}}
\end{equation}

And we have:
\begin{align*}
	\overline{\lambda}(\overline{z}) &= \sum_{i,j}{v_{i} u_{j} \overline{a_{ij}}} \\
	&= v_{I} u_{I} \overline{a_{II}} + v_{I} u_{M} \overline{a_{IM}} + v_{M} u_{I} \overline{a_{MI}} + v_{M} u_{M} \overline{a_{MM}}
\end{align*}

With $\overline{a_{ij}}$ the expected values of the coefficent of the transition matrix. Thus,
\begin{align}
	\overline{\lambda}(\overline{z}) &= v_{I} u_{I} \left[ \overline{f_{1}}(\overline{z}) m s_{0} + (1-m) \overline{s_{I}}(\overline{z}) \right] + v_{I} u_{M} s_{0} \overline{f_{2}}(\overline{z}) \nonumber \\
	&\quad + v_{M} u_{I} m s_{M} + v_{M} u_{M} s_M \\
	\label{eq:dlambda}
	\frac{d\overline{\lambda}(\overline{z})}{d\overline{z}} &= v_{I} u_{I} \left[ \frac{d\overline{f_{1}}(\overline{z})}{d\overline{z}} m s_{0} + (1-m) \frac{d\overline{s_{I}}(\overline{z})}{d\overline{z}} \right] + v_{I} u_{M} s_{0} \frac{d\overline{f_{2}}(\overline{z})}{d\overline{z}}
\end{align}

Because $f_{i}$ and $s_{I}$ are gaussians we can write the population means $\overline{f_{i}}$ and $\overline{s_{I}}$ easily.

\begin{subequations}
	\begin{align}
	\label{eq:meanf}
		\overline{f_{1}}(\overline{z}) &= f_{1}(\theta_{f}) \sqrt{\frac{\omega_{f}}{\omega_{f} + P_{I}}} \exp\left(-\frac{(\overline{z}-\theta_{f})^2}{2(\omega_{f}+P_{I})}\right) \\
		\overline{f_{2}}(\overline{z}) &= f_{2}(\theta_{f}) \sqrt{\frac{\omega_{f}}{\omega_{f} + P_{M}}} \exp\left(-\frac{(\overline{z}-\theta_{f})^2}{2(\omega_{f}+P_{M})}\right) \\
		\overline{s_{I}}(\overline{z}) &= s_{I}(\theta_{s}) \sqrt{\frac{\omega_{s}}{\omega_{s} + P_{I}}} \exp\left(-\frac{(\overline{z}-\theta_{s})^2}{2(\omega_{s}+P_{I})}\right)
	\end{align}
\end{subequations}

Thus we can derive these expression with respect to $\overline{z}$:

\begin{align}
	\label{eq:dfdz}
	\frac{\partial \overline{f_{1}}(\overline{z}) }{ \partial \overline{z} } &= f_{1}(\theta_{f}) \sqrt{\frac{\omega_{f}}{\omega_{f} + P_{I}}} \frac{\partial \exp \left(-\frac{(\overline{z}-\theta_{f})^2}{2(\omega_{f}+P_{I})}\right)}{\partial\overline{z}} \nonumber \\
	&= f_{1}(\theta_{f}) \sqrt{ \frac{\omega_{f}}{ \omega_{f} + P_{I}}}\exp\left(-\frac{(\overline{z}-\theta_{f})^2}{2(\omega_{f}+P_{I})}\right) \frac{\theta_{f} - \overline{z}}{\omega_{f} + P_{I}} \nonumber \\
	&= \overline{f_{1}}(\overline{z}) \frac{\theta_{f} - \overline{z}}{\omega_{f} + P_{I}}
\end{align}

We obtain similar formulas for $\overline{f_{2}}$ and $\overline{s_{I}}$. Plugging \eqref{eq:dfdz} into \eqref{eq:dlambda} we have:

\begin{equation}
	\label{eq:finaldlambda}
	\frac{d\overline{\lambda}(\overline{z})}{d\overline{z}} = v_{I} u_{I} \left[ \frac{\theta_{f} - \overline{z}}{\omega_{f} + P_{I}} m s_{0} + (1-m) \frac{\theta_{s} - \overline{z}}{\omega_{s} + P_{I}} \right] + v_{I} u_{M} s_{0} \frac{\theta_{f} - \overline{z} }{\omega_{f} + P_{M}}
\end{equation}

Using \eqref{eq:finaldlambda} into \eqref{eq:dz} gives us after rearranging:
We have for variations of phenotype, under weak selection:
\begin{equation}
	\label{eq:cstdeltaz}
	\Delta\overline{z} = 
		(\theta_{f} - \overline{z})
		\left[ \frac{ v_{I} u_{I} G_{I} s_{0} m \overline{f_{1}} }{ \lambda (P_{I}+\omega_{f}) }
			+ \frac{ v_{I} u_{M} G_{M} s_{0} \overline{f_{2}} }{ \lambda (P_{M} + \omega_{f}) }
		\right]
		+ (\theta_{s} - \overline{z})
		\left[ \frac{ v_{I} u_{I} G_{I} \overline{s_{I}} (1-m) }{ \lambda (P_{I}+\omega_{s}) }
		\right]
\end{equation}

Within the square brackets, we see weighting average of fecundity and survival. Thus, we define them as $\gamma_{f}$ and $\gamma_{s}$ such as:

\begin{subequations}
	\begin{equation}
	\label{eq:gammaf}
	\gamma_{f} = \frac{v_{I} u_{I} s_{0} m \overline{f_{1}} }{\lambda(P_{I}+\omega_{f})} + \frac{ v_{I} u_{M} \frac{G_{M}}{G_{I}} s_{0} \overline{f_{2}}}{\lambda ( P_{M} + \omega_{f} )}
	\end{equation}
	and
	\begin{equation}
	\label{eq:gammas}
	\gamma_{s} = \frac{ v_{I} u_{I} \overline{s_{I}} (1-m) }{\lambda(P_{I}+\omega_{s})}
	\end{equation}
\end{subequations}

We end up having a simpler expression for $\Delta\overline{z}$ under constant environment:

\begin{align}
	\Delta\overline{z} &= -G_{I} \left[ \gamma_{f}(\overline{z} - \theta_{f}) + \gamma_{s}(\overline{z} - \theta_{s}) \right] \nonumber \\
	\Delta\overline{z} &= - G_{I} \gamma(\overline{z} - \theta_{v})
\end{align}

with
\begin{align}
	\gamma &= \gamma_{f} + \gamma_{s} \\
	\theta_{v} &= \frac{\frac{\gamma_{f}}{\gamma_{s}}\theta_{f} + \theta_{s}}{\frac{\gamma_{f}}{\gamma_{s}} + 1}
\end{align}

\subsection*{Under varying environment}
From \citet{engen_evolution_2011}, we derived equations for mean variation of phenotype on our model.

We supposed an auto-correlated fluctuating environment $\epsilon_{t}$ influencing optimums $\theta_{i}$ such as:
\begin{align}
	\label{eq:epstheta}
\left\{
	\begin{aligned}
		\theta_{i}(t) &= \overline{\theta}_{i} + \alpha_{i}\epsilon_{t}\\
		\epsilon_{t+1} &= (1-\rho)\overline{\epsilon} + \rho\epsilon_{t} + \xi
	\end{aligned}
\right.
\end{align}
with $\alpha_{i}$ the dependence factor of the optimum on the environment, $\rho$ the auto-correlation coefficient of the environment, $\overline{\epsilon}$ the expected environment and $\xi$ a gaussian noise vector with variance $\sigma^{2}_{\xi}$ and mean $0$. We chose $\overline{\epsilon}=0$ to simplify the calculations so that $\epsilon_{t+1} = \rho\epsilon_{t} + \xi$, then:

\begin{align}
	\theta_{i}(t+1) &= \overline{\theta}_{i} + \alpha_{i}\epsilon_{t+1} \nonumber \\
	&= \overline{\theta}_{i} + \alpha_{i}(\rho\epsilon_{t} + \xi) \nonumber \\
	&= \overline{\theta}_{i} + \alpha_{i}\rho(\frac{\theta_{i}(t)-\overline{\theta}_{i}}{\alpha_{i}}) + \alpha_{i}\xi \nonumber \\
	\label{eq:thetait}
	\theta_{i}(t+1) &= \overline{\theta}_{i}(1-\rho) + \rho\theta_{i}(t) + \alpha_{i}\xi
\end{align}

Using the same approach as in a constant environment, under weak selection, we end up having a similar equation than \eqref{eq:cstdeltaz} but with optimum depending on environment:

\begin{equation}
	\label{eq:deltazt}
	\Delta\overline{z}_{t} = 
		(\theta_{f}(t) - \overline{z}_{t})
		\left[ \frac{ v_{I} u_{I} G_{I} s_{0} m \overline{f_{1}} }{ \lambda_{t} (P_{I}+\omega_{f}) }
			+ \frac{ v_{I} u_{M} G_{M} s_{0} \overline{f_{2}} }{ \lambda_{t} (P_{M} + \omega_{f}) }
		\right]
		+ (\theta_{s}(t) - \overline{z}_{t})
		\left[ \frac{ v_{I} u_{I} G_{I} \overline{s_{I}} (1-m) }{ \lambda_{t} (P_{I}+\omega_{s}) }
		\right]
\end{equation}

Plugging \eqref{eq:epstheta} in \eqref{eq:deltazt} we obtain

\begin{subequations}
	\begin{align}
		\Delta\overline{z}_{t} &= 
			(\overline{\theta}_{f} + \alpha_{f}\epsilon_{t} - \overline{z}_{t})
			\left[ \frac{ v_{I} u_{I} G_{I} s_{0} m \overline{f_{1}} }{ \lambda_{t} (P_{I}+\omega_{f}) }
				+ \frac{ v_{I} u_{M} G_{M} s_{0} \overline{f_{2}} }{ \lambda_{t} (P_{M} + \omega_{f}) }
			\right]
			+ (\overline{\theta}_{s} + \alpha_{s}\epsilon_{t} - \overline{z}_{t})
			\left[ \frac{ v_{I} u_{I} G_{I} \overline{s_{I}} (1-m) }{ \lambda_{t} (P_{I}+\omega_{s}) }
			\right] \nonumber \\
	\end{align}
	Defining the same $\gamma_{f}$ and $\gamma_{s}$ as in \eqref{eq:gammaf} and \eqref{eq:gammas}:
	\begin{align}
			\Delta\overline{z}_{t} &= G_{I} \left[(\overline{\theta}_{f} + \alpha_{f}\epsilon_{t} - \overline{z}_{t})
			\gamma_{f}
			+ (\overline{\theta}_{s} + \alpha_{s}\epsilon_{t} - \overline{z}_{t})
			\gamma_{s} \right] \nonumber \\
			&= - G_{I} \left[ \gamma_{f}(\overline{z} - \theta_{f}) + \gamma_{s}(\overline{z} - \theta_{s}) \right] -G_{I} \epsilon_{t} \left( \gamma_{f}\alpha_{f} + \gamma_{s}\alpha_{s} \right) \nonumber \\
			&= \Delta\overline{z} -G_{I} \epsilon_{t} \left( \gamma_{f}\alpha_{f} + \gamma_{s}\alpha_{s} \right)
	\end{align}
	
\end{subequations}